\section{Estudo com Redes Neuronais}
Sendo que o pressuposto das Redes Neuronais é que tenham um bom desempenho, em termos da qualidade da prestação de modelos treinados usando o mesmo, foi a primeira abordagem que decidimos dar ao problema. O maior problema que esperávamos encontrar ao usar este algoritmo era, como seria de esperar, a sua pobre prestação temporal a fazer o treino de cada modelo testado.

\subsection{Estrutura usada}
Pelo facto de termos considerado que o problema não possuía uma complexidade extrema de ser resolvido, apresentando uma dificuldade similar ao do \textit{MNIST}, onde a maior diferença se encontra no facto de neste caso haver uma maior quantidade de classes para identificar, decidimos não usar um modelo de \textit{deep learning}, mas sim uma rede de apenas 3 \textit{layers}:
\begin{itemize}
\item Uma \textit{input layer} com 784 nós.
\item Uma \textit{hidden layer} com 50 nós (foi um dos parâmetros que variamos, mas nos outros estudos, foi sempre este o tamanho usado).
\item Uma \textit{ouput layer} com 24 nós (sendo que a letra "Z" era uma das duas que não podia ser representada estaticamente e correspondendo a um \textit{label} numa das extremidades, não foi contabilizada na \textit{ouput layer}).
\end{itemize}
Para além disso, a inicialização dos \textit{weights} foi feita de forma aleatória em qualquer modelo e os valores dos hiperparâmetros usados por \textit{default} foram:
\begin{itemize}
\item $\lambda = 0$
\item Número de iterações: 1000
\item \textit{Activation Function}: \textit{Logistic sigmoid function}
\item \textit{Batch Size}: 10
\end{itemize}
Os resultados dos \textbf{erros} apresentados em cada um dos próximos tópicos foram obtidos através da formula do \textbf{\textit{Cross-Entropy Loss Function}}, como é possível verificar na equação\ref{error_equation}.
\begin{figure*}[!t]
   \begin{equation}
		E(\theta) = \frac{1}{m}\sum_{i=1}^{m} \sum_{k=1}^{K} [-y_{k}^{i} log((h_{\theta}(x^{i}))_{k}) - (1 - y_{k}^{i}) log(1 - (h_{\theta}(x^{i}))_{k})]
        \label{error_equation}
	\end{equation}
\end{figure*}


\subsection{Estudo da variação do alfa}
A primeira abordagem que decidimos ter ao testar o uso deste algoritmo foi a de variar o valor de alfa, isto é, o valor da \textit{learning rate}, já que este parâmetro dita o tamanho de cada \textit{step} tomado em cada iteração do \textit{gradient descent}, afectando muito directamente a prestação que o modelo treinado apresenta, isto é, tem um papel fundamental no quão boa ou má é feita a generalização do modelo treinado em relação aos dados de \textit{input}.
Desta forma, sendo que no inicio não tínhamos noção de quais os melhores intervalos de alfa para este modelo, decidimos fazer um primeiro treino para valores mais desfasados de alfa. Os erros obtidos nestes modelos podem ser observados no gráfico da figura \ref{fig:nn_initial_error_alfa}, onde se pode perceber que o menor erro foi obtido para um valor de alfa $\alpha = 10^{-5}$, pelo que percebemos que os melhores valores de alfa seriam encontrados em torno desse valor.

\begin{figure}[!t]
\centering
\includegraphics[width=3in]{alpha_error_initial}
\caption{Erro dos modelos obtidos para os valores de alfa $\alpha \in \{10^{-3}, 5*10^{-3}, 10^{-4}, 10^{-5}, 10^{-6}, 10^{-7}\}$.}
\label{fig:nn_initial_error_alfa}
\end{figure}
Sendo assim, decidimos treinar vários modelos para valores de alfa abaixo e acima de $\alpha = 10^{-5}$, mais concretamente, para valores de alfa entre $5*10^{-6}$ e $1.5*10^{-5}$ com intervalos de $10^{-6}$ entre si. Os erros  obtidos para cada um destes modelos podem ser consultados na figura \ref{fig:nn_error_alfa}.

\begin{figure}[!b]
\centering
\includegraphics[width=3in]{alpha_error}
\caption{Erro dos modelos obtidos para os valores de alfa $\alpha \in \{ \alpha \in \!R \mid 5*10^{-6}  \leq \alpha \leq 2*10^{-5} \mid \alpha * 10^{6} \in \!N \}$.}
\label{fig:nn_error_alfa}
\end{figure}
O passo seguinte foi fazermos \textit{retraining} do modelo que teve melhor prestação no passo anterior, ou seja, o que teve menor erro quando submetido aos dados de \textit{cross validation}, neste caso para $\alpha = 1.4^{-5}$. Este passo foi essencial para diminuir ainda mais o erro e ter um modelo com uma ainda melhor prestação. Para isso, usamos o modelo já treinado, isto é, os \textit{weights} obtidos no treino desse modelo e continuamos a treina-lo durante mais 10 000 iterações. Os valores da \textit{cost function} ao longo das 11 000 iterações deste modelo podem ser consulados no gráfico da figura \ref{fig:nn_alpha_cost}. Neste é impossível deixar de reparar que, no momento em que se iniciou o retreino do modelo e em algumas iterações após esse evento, houve um grande aumento do valor da \textit{cost function}. Isto pode ser explicado pelo comportamento conhecido como \textit{Catastrophic interference}, que é tido como o esquecimento total ou parcial daquilo que um modelo tinha aprendido num treino anterior a um novo treino com novos dados ou com novas classes \cite{catastrophic_interference}.
\begin{figure}[!t]
\centering
\includegraphics[width=3in]{alpha_cost}
\caption{\textit{Cost function} para o alfa $\alpha = 1.4 * 10^{-5}$, medida em cada iteração do treino do modelo.}
\label{fig:nn_alpha_cost}
\end{figure}
Em termos de métricas, é possível verificar a performance do modelo perante os dados de teste, consultando os dados da tabela \ref{tab: nn_perforamnce}. Duma forma geral, o modelo apresentou um valor aceitável em qualquer uma das métricas, sendo este valor igual para todas e igual a 97.7\%.

\begin{table}[!t]
\caption{Métricas de performance para $\alpha = 1.4 * 10^{-4}$}
\begin{center}
\begin{tabular}{l c c c c}
Class & Accuracy & Recall & Precision & F1 Score\\ \hline
0 & 1.0 & 1.0 & 0.997 & 0.998\\
1 & 1.0 & 1.0 & 1.0 & 1.0\\
2 & 1.0 & 1.0 & 1.0 & 1.0\\
3 & 1.0 & 1.0 & 0.976 & 0.988\\
4 & 0.996 & 0.996 & 0.965 & 0.981\\
5 & 0.926 & 0.926 & 0.996 & 0.96\\
6 & 1.0 & 1.0 & 0.997 & 0.998\\
7 & 0.978 & 0.978 & 1.0 & 0.989\\
8 & 0.965 & 0.965 & 0.962 & 0.964\\
10 & 0.984 & 0.984 & 0.95 & 0.967\\
11 & 1.0 & 1.0 & 0.996 & 0.998\\
12 & 0.99 & 0.99 & 0.922 & 0.955\\
13 & 0.911 & 0.911 & 0.988 & 0.948\\
14 & 0.993 & 0.993 & 0.978 & 0.985\\
15 & 0.993 & 0.993 & 1.0 & 0.996\\
16 & 0.948 & 0.948 & 0.979 & 0.963\\
17 & 0.958 & 0.958 & 0.989 & 0.974\\
18 & 0.949 & 0.949 & 0.986 & 0.967\\
19 & 0.984 & 0.984 & 0.953 & 0.968\\
20 & 0.961 & 0.961 & 0.976 & 0.969\\
21 & 0.972 & 0.972 & 0.965 & 0.969\\
22 & 0.964 & 0.964 & 0.964 & 0.964\\
23 & 0.973 & 0.973 & 0.943 & 0.958\\
24 & 0.997 & 0.997 & 0.971 & 0.984\\
\hline
Macro Average & 0.977 & 0.977 & 0.977 & 0.977\\
\end{tabular}
\label{tab: nn_perforamnce}
\end{center}
\end{table}

Quanto aos valores da matriz de confusão deste modelo, com os dados de teste, podem ser consultados na tabela \ref{tab: nn_confusion_matrix}. Nela, pode-se constatar que, de uma forma geral, o modelo praticamente adivinhou correctamente todas as classes.

\begin{table*}[!t]
\caption{Matriz de confusão para $\alpha = 1.4 * 10^{-4}$}
\begin{center}
\setlength{\tabcolsep}{0.5em}
\begin{tabular}{l l |c c c c c c c c c c c c c c c c c c c c c c c c }
{} & {} & \multicolumn{24}{c}{Actual Class}\\
{} & Class&0&1&2&3&4&5&6&7&8&10&11&12&13&14&15&16&17&18&19&20&21&22&23&24\\
\hline
\multirow{24}{*}{\rotatebox[origin=c]{90}{ Predicted Class}}&0&301&0&0&0&0&0&0&0&0&0&0&0&0&0&0&0&0&0&0&0&0&0&0&0\\
&1&0&291&0&0&0&0&0&0&0&0&0&0&0&0&0&0&0&0&0&0&0&0&0&0\\
&2&0&0&281&0&0&0&0&0&0&0&0&0&0&0&0&0&0&0&0&0&0&0&0&0\\
&3&0&0&0&282&0&0&0&0&0&0&0&0&0&0&0&0&0&0&0&0&0&0&0&0\\
&4&0&0&0&0&277&0&0&0&0&0&0&0&0&0&0&0&0&1&0&0&0&0&0&0\\
&5&0&0&0&0&0&274&0&0&0&5&0&0&0&0&0&0&0&0&5&0&5&0&7&0\\
&6&0&0&0&0&0&0&305&0&0&0&0&0&0&0&0&0&0&0&0&0&0&0&0&0\\
&7&0&0&0&0&0&0&1&307&6&0&0&0&0&0&0&0&0&0&0&0&0&0&0&0\\
&8&1&0&0&0&0&0&0&0&278&0&0&0&0&0&0&0&0&0&9&0&0&0&0&0\\
&10&0&0&0&0&0&0&0&0&4&246&0&0&0&0&0&0&0&0&0&0&0&0&0&0\\
&11&0&0&0&0&0&0&0&0&0&0&274&0&0&0&0&0&0&0&0&0&0&0&0&0\\
&12&0&0&0&0&0&0&0&0&0&0&0&308&0&0&0&1&0&2&0&0&0&0&0&0\\
&13&0&0&0&0&0&0&0&0&0&0&0&22&256&0&0&3&0&0&0&0&0&0&0&0\\
&14&0&0&0&0&0&0&0&0&0&0&0&0&2&268&0&0&0&0&0&0&0&0&0&0\\
&15&0&0&0&0&0&0&0&0&0&0&0&0&0&0&270&1&1&0&0&0&0&0&0&0\\
&16&0&0&0&5&0&0&0&0&0&0&0&4&0&6&0&275&0&0&0&0&0&0&0&0\\
&17&0&0&0&0&0&0&0&0&0&0&0&0&0&0&0&0&277&0&0&6&0&6&0&0\\
&18&0&0&0&0&10&0&0&0&1&0&0&0&1&0&0&1&0&279&0&0&0&0&0&2\\
&19&0&0&0&0&0&0&0&0&0&0&1&0&0&0&0&0&0&0&304&0&0&0&4&0\\
&20&0&0&0&1&0&0&0&0&0&8&0&0&0&0&0&0&1&0&0&248&0&0&0&0\\
&21&0&0&0&0&0&0&0&0&0&0&0&0&0&0&0&0&0&0&0&0&279&5&0&3\\
&22&0&0&0&0&0&1&0&0&0&0&0&0&0&0&0&0&0&0&0&0&0&295&6&4\\
&23&0&0&0&1&0&0&0&0&0&0&0&0&0&0&0&0&1&0&1&0&5&0&283&0\\
&24&0&0&0&0&0&0&0&0&0&0&0&0&0&0&0&0&0&1&0&0&0&0&0&306\\
\end{tabular}
\label{tab: nn_confusion_matrix}
\end{center}
\end{table*}

\subsection{Estudo da variação do tamanho da Hidden Layer}
Decidimos também, como parte do nosso estudo, verificar o comportamento deste algoritmo para a variação do tamanho da \textit{hidden layer}, e verificar como o maior ou menor tamanho desta afectava a prestação do nosso modelo, bem como a complexidade computacional do treino do mesmo. Para isso, usamos o valor do melhor alfa obtido no ponto anterior e variámos o valor do número de nós da \textit{hidden layer} de 5 em 5, entre 5 e 70. Assim sendo, obtivemos o gráfico dos erros de treino e de validação apresentado da figura \ref{fig:hiddent_error}.

\begin{figure}[!b]
\centering
\includegraphics[width=3in]{hidden_layer1_error}
\caption{Erro dos modelos obtidos para $n \in \{ n \in \!N \mid 5  \leq n \leq 70 \mid n / 5 \in \!N \}$, onde $n$ representa o número de nós da \textit{hidden layer}.}
\label{fig:hiddent_error}
\end{figure}
Dos resultados obtidos, é-nos possível entender que o valor do erro para os dados de validação começa a estagnar para um número de nós acima de 40, inclusive. Uma possível explicação para este comportamento acontecer pode-se basear no facto de que, para um número de nós acima de 40, o custo do treino já praticamente convergiu. Como se pode perceber no gráfico da figura \ref{fig:hiddent_cost}, para valores muito pequenos do número de nós, para 1000 iterações, a "rapidez" com que o custo converge é muito menor do que para valores mais elevados, sendo que para estes últimos, nas ultimas iterações, os custos são muito parecidos de modelo para modelo.
\begin{figure}[!t]
\centering
\includegraphics[width=3in]{hidden_layer1_cost}
\caption{\textit{Cost function} para $n \in \{ n \in \!N \mid 5  \leq n \leq 70 \mid n / 5 \in \!N \}$, onde $n$ representa o número de nós da \textit{hidden layer}, medida em cada iteração do treino de cada modelo.}
\label{fig:hiddent_cost}
\end{figure}
Fizemos também um estudo da complexidade temporal associado ao tamanho da \textit{hidden layer}, já que este está directamente relacionado com a complexidade computacional do treino, já que quantos mais nós houver nesta camada, maior será o número de cálculos matriciais feitos. Como demonstrado no gráfico da figura \ref{fig:hiddent_time}, conclui-se que o tempo de execução varia, duma forma geral, linearmente de acordo com o número de nós. Por conseguinte, num caso de estudo onde se privilegia o tempo com que o treino é executado em detrimento de alguma qualidade nas previsões feitas pelo modelo treinado, possivelmente seria aconselhável utilizar um número de nós mais reduzido, como por exemplo 40, obtendo-se um erro aceitável, mas claro, não tão baixo como seria se se usassem mais nós.
\begin{figure}[!t]
\centering
\includegraphics[width=3in]{hidden_layer1_time}
\caption{Tempo de execução para $n \in \{ n \in \!N \mid 5  \leq n \leq 70 \mid n / 5 \in \!N \}$, onde $n$ representa o número de nós da \textit{hidden layer}.}
\label{fig:hiddent_time}
\end{figure}

\subsection{Estudo da variação do número de iterações}
Ao obtermos o melhor valor de alfa, constatamos ao analisar o gráfico do custo de treino do melhor modelo ao longo das várias iterações que a \textit{cost function} começava a convergia muito antes de completar 1000 iterações. Sendo assim, decidimos fazer um estudo da forma como o número de iterações iria afectar a performance do modelo. Na figura \ref{fig:num_iterations_error} é possível analisar o gráfico da evolução do erro de acordo com o número de iterações, para alfa $\alpha = 1.4 * 10^{-5}$. Neste, pode-se constatar que para mais de 500 iterações, o valor do erro começa a ser parecido entre as várias iterações. Sendo que o número de iterações está intimamente relacionado com a complexidade computacional do algoritmo, como se pode verificar no gráfico da figura \ref{fig:num_iterations_time},   chegamos à conclusão de que é possível obter modelos com uma performance aceitável em menos tempo, se executados durante menos iterações. 

\begin{figure}[!t]
\centering
\includegraphics[width=3in]{num_iterations_error}
\caption{Erro dos modelos obtidos para $i \in \{ i \in \!N \mid 100 \leq i \leq 1000 \mid i/100 \in \!N \}$, onde $i$ representa o número de iterações.}
\label{fig:num_iterations_error}
\end{figure}

\begin{figure}[!t]
\centering
\includegraphics[width=3in]{num_iterations_time}
\caption{Tempo de execução para $i \in \{ i \in \!N \mid 100 \leq i \leq 1000 \mid i/100 \in \!N \}$, onde $i$ representa o número de iterações.}
\label{fig:num_iterations_time}
\end{figure}
De acordo com vários estudos feitos, o método de \textit{early stopping} é já usado em muitos estudos e projectos que usam redes neuronais, dado o facto da complexidade computacional ser muito mais baixa e de automatizar o processo de seleccionar o melhor modelo de acordo com o número de iterações necessárias para a curva da \textit{cost function} começa a convergir \cite{early_stopping}.
Desta forma, decidimos experimentar treinar um modelo nas mesmas condições, mas usando este método, de maneira a verificar se conseguia-mos obter resultados aceitáveis. Neste caso, usamos um \textit{stopping criteria} inverso do terceiro descrito no estudo \cite{early_stopping}, isto é, o treino acaba se, passado um determinado número de iterações consecutivas, $nc$, o valor da função de custo não melhorar mais do que um determinado valor predefinido, $t$ (tolerância). Para $nc = 30$, $t = 10^{-4}$ e $\alpha = 1.4 * 10^{-5}$, o modelo treinou durante 778 iterações e a performance obtida perante os dados de teste pode ser consultada na tabela \ref{tab: nn_early_stopping}. Como seria de esperar, a performance não foi tão boa como para o modelo que usou 1000 iterações para o mesmo valor de alfa, mas pode ser aceitável em certos casos de uso, ainda para mais tendo em conta que o tempo de execução é muito mais reduzido do que fazer a totalidade de 1000 iterações. 

\begin{table}[!t]
\caption{Métricas de performance para um modelo cujo treino usou \textit{early stopping} com $nc = 30$, $t = 10^{-4}$ e $\alpha = 1.4 * 10^{-5}$ e foi executado durante 778 iterações}
\begin{center}
\begin{tabular}{l c c c c}
Class & Accuracy & Recall & Precision & F1 Score\\ \hline
0 & 1.0 & 1.0 & 0.974 & 0.987\\
1 & 0.911 & 0.911 & 1.0 & 0.953\\
2 & 0.989 & 0.989 & 0.952 & 0.97\\
3 & 1.0 & 1.0 & 0.959 & 0.979\\
4 & 0.968 & 0.968 & 0.957 & 0.962\\
5 & 0.943 & 0.943 & 0.969 & 0.955\\
6 & 0.918 & 0.918 & 0.982 & 0.949\\
7 & 0.917 & 0.917 & 0.886 & 0.901\\
8 & 0.979 & 0.979 & 0.979 & 0.979\\
10 & 0.912 & 0.912 & 0.954 & 0.933\\
11 & 0.978 & 0.978 & 0.931 & 0.954\\
12 & 0.707 & 0.707 & 0.952 & 0.812\\
13 & 0.701 & 0.701 & 0.99 & 0.821\\
14 & 0.878 & 0.878 & 0.967 & 0.92\\
15 & 0.926 & 0.926 & 0.984 & 0.955\\
16 & 0.997 & 0.997 & 0.963 & 0.98\\
17 & 0.779 & 0.779 & 0.996 & 0.874\\
18 & 0.959 & 0.959 & 0.597 & 0.736\\
19 & 0.968 & 0.968 & 0.824 & 0.89\\
20 & 0.938 & 0.938 & 0.864 & 0.9\\
21 & 0.927 & 0.927 & 0.818 & 0.869\\
22 & 0.886 & 0.886 & 0.858 & 0.871\\
23 & 0.876 & 0.876 & 0.985 & 0.927\\
24 & 0.935 & 0.935 & 0.96 & 0.947\\
\hline
Macro Average & 0.916 & 0.916 & 0.929 & 0.918\\
\end{tabular}
\label{tab: nn_early_stopping}
\end{center}
\end{table}
\subsection{Estudo da variação do tamanho da batch}
Por ultimo, decidimos analisar como o tamanho da \textit{batch} poderia influenciar a qualidade do modelo obtido. Para isso, usamos o melhor alfa já anteriormente obtido e treinamos os modelos durante 1000 iterações. De acordo com estudos já feitos nesta área, é possível aumentar o tamanho da \textit{batch} durante o processo de treino ao invés da diminuição da \textit{learning rate} e, mesmo assim, obter resultados similares, mas num menor intervalo de tempo, já que um maior tamanho da \textit{batch} implica menos actualizações dos parâmetros de treino \cite{batch_size_increase}. Contudo, nós neste caso, e por uma questão de facilidade e limitações da ferramenta que usamos, o \textit{scikit-learn}, fizemos um estudo para o treino de vários modelos com tamanhos de \textit{batch} distintos entre si e constantes durante o treino, para um mesmo valor de alfa. O valor dos erros obtidos podem ser consultados no gráfico da figura \ref{fig:batch_size_error}.
\begin{figure}[!t]
\centering
\includegraphics[width=3in]{batch_size_error}
\caption{Erro dos modelos obtidos para $b \in \{ b \in \!N \mid 8 \leq b \leq 256 \mid \sqrt{b} \in \!N \}$, onde $b$ representa o tamanho da \textit{batch} usada.}
\label{fig:batch_size_error}
\end{figure}
Numa primeira análise destes resultados, acha-mos estranho, já que esperávamos que o erro baixasse e não subisse, pelo menos para os dados de treino, ao aumentarmos o tamanho da \textit{batch}. Contudo, ao analisarmos o gráfico da função de custo, verificamos que para um tamanho de \textit{batch} mais elevado, a função convergia mais lentamente, como se pode verificar no gráfico da figura \ref{fig:batch_size_cost}. Sendo assim, teríamos duas hipóteses: aumentar o número de iterações, o que não era pretendido neste caso, porque queríamos verificar se era possível obter um menor tempo de treino do que com \textit{batchs} mais pequenas, ou aumentar o valor da \textit{learning rate}, para um valor fixo do tamanho da \textit{batch}, de forma a que a função de custo convergi-se mais rapidamente, tendo sido este o procedimento tomado e cujos resultados podem ser consultados no gráfico de erros da figura \ref{fig:batch_size_error256} e na tabela de métricas do modelo com menor erro (no \textit{test dataset}) \ref{tab: batch_alpha256}. 

\begin{figure}[!t]
\centering
\includegraphics[width=3in]{batch_size_cost}
\caption{\textit{Cost function} para $b \in \{ b \in \!N \mid 8 \leq b \leq 256 \mid \sqrt{b} \in \!N \}$, onde $b$ representa o tamanho da \textit{batch} usada.}
\label{fig:batch_size_cost}
\end{figure}

\begin{figure}[!t]
\centering
\includegraphics[width=3in]{alpha_error256}
\caption{Erro dos modelos obtidos para os valores de alfa $\alpha \in \{ \alpha \in \!R \mid 10^{-4}  \leq \alpha \leq 10^{-3} \mid \alpha * 10^{4} \in \!N \} \cup \{ \alpha \in \!R \mid 9 * 10^{-5}  \leq \alpha \leq 10^{-5} \mid \alpha * 10^{5} \in \!N \}$ e para \textit{batch size} $b = 256$.}
\label{fig:batch_size_error256}
\end{figure}

\begin{table}[!t]
\caption{Métricas de performance para \textit{batch size} $b = 256$ e alfa $\alpha = 9 * 10^{-5}$}
\begin{center}
\begin{tabular}{l c c c c}
Class & Accuracy & Recall & Precision & F1 Score\\ \hline
0 & 1.0 & 1.0 & 0.98 & 0.99\\
1 & 0.979 & 0.979 & 0.966 & 0.973\\
2 & 0.996 & 0.996 & 1.0 & 0.998\\
3 & 1.0 & 1.0 & 0.986 & 0.993\\
4 & 0.917 & 0.917 & 0.992 & 0.953\\
5 & 0.932 & 0.932 & 0.962 & 0.947\\
6 & 0.954 & 0.954 & 0.942 & 0.948\\
7 & 0.949 & 0.949 & 0.99 & 0.969\\
8 & 0.962 & 0.962 & 0.982 & 0.972\\
10 & 0.948 & 0.948 & 0.868 & 0.906\\
11 & 0.945 & 0.945 & 0.827 & 0.882\\
12 & 0.932 & 0.932 & 0.924 & 0.928\\
13 & 0.936 & 0.936 & 0.916 & 0.926\\
14 & 0.981 & 0.981 & 0.996 & 0.989\\
15 & 0.908 & 0.908 & 0.888 & 0.898\\
16 & 0.938 & 0.938 & 0.925 & 0.932\\
17 & 0.893 & 0.893 & 0.819 & 0.854\\
18 & 0.874 & 0.874 & 0.918 & 0.895\\
19 & 0.919 & 0.919 & 0.937 & 0.928\\
20 & 0.919 & 0.919 & 0.919 & 0.919\\
21 & 0.857 & 0.857 & 0.908 & 0.882\\
22 & 0.83 & 0.83 & 0.951 & 0.887\\
23 & 0.893 & 0.893 & 0.935 & 0.914\\
24 & 0.98 & 0.98 & 0.929 & 0.954\\
\hline
Macro Average & 0.935 & 0.935 & 0.936 & 0.935\\
\end{tabular}
\label{tab: batch_alpha256}
\end{center}
\end{table}

Apesar de não se ter obtido uma tão boa performance como quando utilizado um tamanho de \textit{batch} mais pequeno (97.7\% de \textit{Accuracy} vs. 93.5\% neste caso), é de notar que a performance é elevada e pode ser aceitável em muitos estudos, onde seja dada mais importância à rapidez com que o modelo é treinado. Como se pode verificar no gráfico da figura \ref{fig:batch_size_time}, a rapidez com que o treino é feito diminui substancialmente com o aumento do \textit{batch size}.

\begin{figure}[!t]
\centering
\includegraphics[width=3in]{batch_size_time}
\caption{Tempo de execução para $b \in \{ b \in \!N \mid 8 \leq b \leq 256 \mid \sqrt{b} \in \!N \}$, onde $b$ representa o tamanho da \textit{batch} usada.}
\label{fig:batch_size_time}
\end{figure}
\subsection{Conclusões}
Um ponto importante a destacar é que não fizemos qualquer estudo da variação do lambda \footnote{parâmetro de regularização}. Isto deve-se ao facto de não ter havido uma discrepância muito notável entre os erros de treino e de \textit{cross validation}, ou seja, não obtivemos qualquer tipo de \textit{overfit}. Este problema poderá advir da \textit{data augmentation} que foi feita aos dados originais, sendo que, ao os novos dados estarem distribuídos aleatoriamente pelos três \textit{datasets}, tenha provocado que os modelos treinados tiveram contacto com todas as imagens originais, não existindo qualquer novidade quando foi submetido aos dados de validação e de teste.
Outro problema a apontar ao uso deste algoritmo é o facto da computação ser muito pesada, ainda para mais se feita com recurso a um \textit{CPU} ao invés dum \textit{GPU}. A principal diferença entre usar um \textit{CPU} e um \textit{GPU} está no facto deste ultimo usar   uma muito menor quantidade de núcleos que o primeiro. Sendo que as computações mais exigente das Redes Neuronais são cálculos matriciais, que podem ser paralelizados, um \textit{GPU} tem a capacidade de paralelizar uma muito maior quantidade de cálculos em simultâneo que um \textit{CPU} pelo simples facto de possuir muitos mais núcleos para os fazer. Existem também outras razões para esses cálculos serem executados mais rapidamente em \textit{GPUs}, como a maior largura de banda e a maior dimensão e rapidez que os seus registos de memória possuem \cite{cpu_vs_gpu}. Sendo que usamos o \textit{scikit-learn} para fazer o estudo e este apenas executa as computações em \textit{CPU}, foi extremamente penoso proceder ao treino dos vários modelos apresentados. Chegamos por isso à conclusão de que, para este algoritmo, a resposta seja a utilização duma biblioteca como o \textit{TensorFlow}, que permite a utilização do \textit{GPU} do computador para treinar os modelos de Redes Neuronais.
Em termos dos valores dos parâmetros a usar neste algoritmo para este problema, temos duas visões, que se baseiam no \textit{trade off} que tem de ser feito entre a complexidade temporal e a qualidade dos resultados obtidos:
\begin{itemize}
\item Por um lado, se quisermos utilizar um modelo para um projecto onde seja privilegiada a qualidade dos resultados obtidos em detrimento do custo temporal que o treino do modelo necessita, então é recomendada a utilização dum valor de alfa reduzido, um \textit{batch size} mais pequeno, um número de iterações elevado, sem \textit{early stop} e uma \textit{hidden layer} com um tamanho razoável (não muito elevado, para não ocorrer \textit{overfit}).
\item Por outro lado, se pretendemos usar o modelo num projecto onde seja privilegiado custo temporal e onde é aceitável obter resultados menos satisfatórios, chegamos à conclusão de que a utilização dum alfa com um valor mais elevado, com um tamanho de \textit{batch} grande, com um número de iterações determinadas por uma regra de \textit{early stop} e uma \textit{hidden layer} de tamanho médio, levam à obtenção dum modelo que não possui a melhor performance possível, mas razoável e cujo tempo de treino é aceitável.
\end{itemize}
Claro que estas conclusões, apesar de para este problema em específico, podem ser generalizadas para outros problemas do mesmo género.
